\unnumberedChapter[consistency]{Consistency Among Architectural Views}
Regarding the consistency among the architectural views, we can say that the views do not present mayor incoherences considering the small amount of time we add to produce them and the early stage of the project. Nevertheless, more attention has been accorded to the view that are less prone to change with the development of the project. In fact, we considered that trying to produce a detailed class diagram or even trying to predict what would be the different classes, interfaces and packages of the future application was not relevant and almost infeasible for the moment. \\

We still consider that the purposes of the logical view and the development view have been attained. For the logical view, we can see that the model of the domain contain much more details that the class diagram itself. Even if their structure is somewhat similar, the modeling of the domain is a better tool at the beginning of the project than a class diagram since it will provide a common language to both the end-users and the development team and ensure that everyone is on the same page comparatively to a class diagram that can often be challenging to understand and to present properly into a report. For the development view, its main purpose is to ease the separation of the workload among the developers. Although we do not have an exhaustive listing of all the code components to be implemented, we are still able to divide the work properly. With the advancement of the project, we will have a better idea of what should appears into those views and we will update them in consequence. \\

Another incoherence may be related to usage of different nomenclature of the architecture components. One the most egregious difference is probably the interchanging usage of the words items and powerups. \\

The last incoherence we can observe is between the process view showing the navigation through the screens and the development diagram that mention the presence of a credit screen and a loading screen whereas the process view does not show those two elements. Inversely, the process view seems to suggest the existence of an pause screen which is not mentioned in the development view.   
