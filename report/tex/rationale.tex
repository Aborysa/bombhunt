\section{Architectural Rationale}
We already went into slight detail on how the patterns will benefit us previously in the document, but we will go more in depth here.
\\More members in this group has already had a lot of experience creating different software from before, and a lot of the time, we are forced to pass around a lot of objects, so they can be referenced deeper in the object tree. Say for instance some Logger that will keep track of what has been going on in the game. If that had to be passed around all the time to be called when something needs logging, it would end up quite messy. 
\\Since it's a logger as well, it would only make sense to have one instance of it at all time during the game, which is when this pattern also comes in handy and guarantees us one single instance of it at all time.
\\ To make the code more tidier and keep a more organized structure on how we will let the game know about states and events, we will use the observer pattern to accomplish this.
\\ When we first started researching patterns, we were quite fond of the ECS, as it allowed us to be more dynamic and flexible when creating entities for our game. It also makes it quite easy to fulfil the Modifiability attribute of the game, that specified it should be easy to change or configure in order to add or modify functionality. This is definitely the pattern we're the most exited about.
\\ Since our secondary attribute was usability, and since MVC is such a widely used standard, we knew if we manage to implement this well, we will fulfil this attribute with an easy to use interface for the user that can interact with all the underlying logic.
\\The actual game logic won't be affected by the MVC, but rather referring to the game world as the model, and the controls and interface and controls as the MVC. As well as the menu screens and interactions we will have there.
\\ If we manage to set up P2P, it will be a lot easier to handle all the data that is usually passed around in multiplayer games. Since all data will be passed between the players themselves, and since there will be quite limited how many players are on at a time, it will be a lot easier for us to develop it. Initially we think Google Play Service will be the solution that will help us.
