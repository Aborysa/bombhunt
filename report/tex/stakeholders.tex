\unnumberedChapter[stakeholders]{Stakeholders and Concerns}
This section outlines the main stakeholders of the system and their concerns related to the software architecture.

\unnumberedSection[lecturer]{Lecturer}
The lecturer for this course is interested to see how the students perform, and as well what they deliver in the final version of the game. To make sure this can be accomplished as easy as possible for the lecturer, we will make sure to make our architecture as easy as possible to understand, and as well look through, by documenting everything as good as possible. This will make it easier to evaluate us for the final grade, and hopefully make it more enjoyable to read up on our architecture and test our game.

\unnumberedSection[other groups]{Other groups assigned to reviewing our work}
During this course, other groups in the same course will be assigned to review our documentation and the architecture itself. To make it easier for the other groups to review our work, we will make sure our views and tables are reflecting our game as good as possible. This way the architectures will be as clear as possible, and should be very easy to understand, even if you had no experience with our chosen patterns and code.

\unnumberedSection[the group]{Project participants}
We, the developers of the project, are concerned in being able to deliver the architecture we visioned. As our selected architecture is quite ambigious, we're required to be quite motivated to deliver what we want to make. As we are motivated to get a high as possible grade in this course, this shouldn't be an issue.

\unnumberedSection[players]{Players of the game}
The actual players of the game, or the end users, are concerned with how good the game is to play. They're not as concerned with how the game is developed or structured, but more interested with the features and functionality. This means our logical view has to be on point, and as well make sure it's well represented within the game itself.
