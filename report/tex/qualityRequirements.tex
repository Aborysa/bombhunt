\unnumberedChapter[QAReq]{Quality Requirements}

%Write scenarios for the most relevant quality attributes (modifiability, usability, availability, performance and interoperability)

Quality attributes usually have an important impact on the overall architecture because their realization often imply the modification of many part of the application whereas functional requirements are usually more located. Moreover, quality attributes are often more difficult to assert due to their subjective meaning. Per example, some people may have different conceptions of what is defined as a secure application. The purpose of this chapter is to determine some thresholds that will uncover this subjectivity. \\

To specify the different quality attributes, a scenario approach will be used as the one presented in \cite[Chapter 4, p.68-69]{bass2013}. According to this method, a quality attribute can be made unambiguous and testable by specifying the following 6 characteristics:
\begin{itemize}
  \item Stimulus
  \item Stimulus source
  \item Response
  \item Response measure
  \item Environment
  \item Artifact
\end{itemize}

\unnumberedSection[modifiability]{Modifiability}

\QAReq{M1}
{Quality Requirement Scenario - Modifying existing graphics}
{modGraphics}
{Developer}
{Wishes to modify existing graphics}
{Graphical assets}
{Design Time}
{Graphic updated}
{The graphics are updated and seamlessly integrated into the game in less than half an hour.}

\QAReq{M2}
{Quality Requirement Scenario - Modifying existing audio components}
{modAudio}
{Developer}
{Wishes to modify existing audio components}
{Audio assets}
{Design Time}
{Audio components updated}
{The audio components is updated and seamlessly integrated into the game in less than half an hour.}

\QAReq{M3}
{Quality Requirement Scenario - Modifying logic of an existing component}
{modLogic}
{Developer}
{Wishes to modify the logic of an existing component}
{Code}
{Design Time}
{Changes performed, functional tests passed and integration tests passed}
{Change made in less than half an hour without affecting the component’s interface or other parts of the software.}

\QAReq{M4}
{Quality Requirement Scenario - Adding new powerup item}
{addItem}
{Developer}
{Wishes to add a new powerup item}
{Code}
{Design Time}
{Addition performed, functional tests passed and integration tests passed}
{Change made in less than two hours without affecting the item’s interface or other parts of the software.}

\QAReq{M5}
{Quality Requirement Scenario - Adding new character}
{addCharacter}
{Developer}
{Wishes to add a new character}
{Code and assets}
{Design Time}
{Addition performed, functional tests passed and integration tests passed. The skills of the new character have to be balanced to match the average capacities of the existing characters.}
{Change made in less than three hours without affecting the character’s interface or other parts of the software.}

\QAReq{M6}
{Quality Requirement Scenario - Adding new map}
{addMap}
{Developer}
{Wishes to add a new map}
{Code and assets}
{Design Time}
{Addition performed, functional tests passed and integration tests passed.}
{Change made in less than seven hours without affecting the current map creation process. The time account for the creation of the map.}

\QAReq{M7}
{Quality Requirement Scenario - Adding new achievement}
{addAchievement}
{Developer}
{Wishes to add a new achievement}
{Code and assets}
{Design Time}
{Addition performed, functional tests passed and integration tests passed. The new achievement attribution process receive all the needed information from the database connection.}
{Change made in less than three hours without affecting the achievement’s interface or other parts of the software. This consider that all the information need was already available for the attribution.}

\QAReq{M8}
{Quality Requirement Scenario - Modifying score calculation algorithm}
{modScore}
{Developer}
{Wishes to modify the scoring algorithm}
{Code}
{Design Time}
{Addition performed, functional tests passed and integration tests passed. Users reranked and leaderboard updated.}
{Change made in less than five hours without affecting the structure of the database or other parts of the software. The time ignore the time needed to reprocess the user ranking.}

\QAReq{M9}
{Quality Requirement Scenario - Modifying database structure}
{modData}
{Developer}
{Wishes to modify the database structure}
{Code and databases}
{Design Time}
{Addition performed, functional tests passed and integration tests passed. All the requests made to the database processed correctly.}
{Change made in less than seven hours without affecting the parts of the software that depends on the database.}

\unnumberedSection[usability]{Usability}

\QAReq{U1}
{Quality Requirement Scenario - First-time play}
{useFirstTime}
{User}
{Wants to play for the first time}
{System}
{Runtime}
{User is introduced to controls and mechanics within the game through
a short and simple textual tutorial.}
{Finished the tutorial within 3 minutes and has a basic understanding of the game components and its controls.}

\QAReq{U2}
{Quality Requirement Scenario - Introduction to multiplayer mode}
{useFirstMulti}
{User}
{Wants to join an online multiplayer match for the first time}
{System}
{Runtime}
{User enters in communication with the server, reaches the lobby and gets assigned to a match.}
{The user does not have to modify any internet communication settings to get a connection to the server as long as a valid internet connection is available. The user do not remain unassigned for an excessive amount of time (less than 2 minutes). In case of failure to join a match, the user receives a message mentioning the reason why no match as been found. User should be able to start playing within 3 minutes on average.}

\QAReq{U3}
{Quality Requirement Scenario - Modify game through settings}
{useSettings}
{User}
{Wants to modify a setting of the game}
{System}
{Runtime}
{User finds the setting and successfully modifies it.}
{Setting is found without any navigation-errors in 15 seconds or less.}


\unnumberedSection[availability]{Availability}
\unnumberedSection[performancey]{Performance}
\unnumberedSection[interoperability]{Interoperability}
