\unnumberedChapter[asr]{Architectural Drivers / Architectural Significant Requirements}
	This section outlines the main drivers that most affect the system architecture.

	\unnumberedSection[funcReq]{Functional Requirements}
	\unnumberedSubsection[funcReq:multiplayer]{Online Mulitplayer}
	The most important functional requirement is the ability to play mulitplayer online with other	players. All our requirements follow after this core requirement. It is critical that the data and logic is structured in such a way that it is easy to synchronize data, state and events in such a way that we minimize the possibility for errors. In case of errors due to networking issues, we should be able to recover. The most sensible way to achieve this would be to mark all entities 	that should be synchronized with one or multiple components containing the network state of an entity, like what client created the entity and what simulation frame the entity is supposed to be at a given time.

	\unnumberedSubsection[funcReq:classes-and-powerups]{Different Player Classes and Powerups}
	The player should be able to select a character from a set of characters all with differing abilities. The abilities can be altered during gameplay by acquiring powerups. It is therefore very important that components are made in such a way that it is easy to create modify abilities in the middle of the game.

	\unnumberedSection[qaReq]{Quality Requirements}
	\unnumberedSubsection[qaReq:modifiability]{Modifiability}
	Modifiability is our primary QR. It is very important that we structure our game in good. For instance it should be possible to implement new game modes and have them work without having to modify the core engine. Ideally it should be as simple as implementing a predefined interface that defines a set of rules for the game mode and is able to perform meaningful actions with as little knowledge of the other modules of the system. Modifiability is also important to let us rapidly prototype new features that might be interesting without too much overhead for it to be feasible.

	\unnumberedSubsection[qaReq:usability]{Usability}
	Our game should be easy to use, we want a user to be able to just jump straight into the action with as little introduction as possible. The user should always be aware of what state the game is in and the controls and menus should be as minimally bloated as possible.

	\unnumberedSubsection[qaReq:performance]{Performance}
	It is very important that the game is performant for the stability of the multiplayer. If one client isn't able to perform the required ticks per second to be able to simulate	and keep the state synchronized between the clients then there is going to be a very big issue. It is therefore important that the game loops are used wisely and the game is able to detect whether or not it is able to keep up and automatically able to disable unimportant features like particle effects.

	\unnumberedSubsection[qaReq:interoperability]{Interoperability}
	Interoperability is important to achieve good and reliable mulitplayer experience across android clients made by different hardware manufacturers and across different local area networks that might be configured differently. Normally p2p applications have to do a hole punchthrough using a broker server in order to establish a reliable p2p connection. By using Google Play Game Services we avoid having to deal with initiating connections between players ourselves, because the google services handles it for us.

	\unnumberedSection[businessReq]{Business Requirements}
	We want to create an overall engaging experience with interesting and fun gameplay elements. The game should have competetive aspects	like highscores, leaderboards and achivements.
