\unnumberedChapter[issues]{Issues}
The goal of this chapter is to document the different issues that we faced during the creation of the following document and the development of the project in general.

\unnumberedSection[firstExperienceECS]{First Experience with ECS Pattern}
Even thought half of the best members were really familiar with the \gls{ecs} pattern, it was a first time for the reset of the team. A lot of time has been spent on learning on the pattern works instead of developing features. This contributed too a huge reduction of effective time devoted to the project.

\unnumberedSection[unreliableGPGS]{Unreliable Google Play Service Messaging}

\unnumberedSection[poorPeerReview]{Poor Peer-Review Process}
Early stage of the project, we have been too loose on the approving of pull requests. Ask for refactoring, no generally accepted coding standards. Established new peer-review process within 24h and testing before accepting. DONT TRUST ANYBODY.

\unnumberedSection[tooManyDeliverable]{Too Many Deliverable}
Having so many deliverables made the whole project even more difficult to manage. Pushing some members to be fully in charge of specific deliverable without time for peer-reviewing

\unnumberedSection[highExpectations]{High Expectations}
The team members had unrealistic expectations about the final product given the time provided to do the project. Would not have complained if this project would be deliverable at the end of semester at the same time of the exam...

for our project we were a bit too ambitious with what we wanted to create, which resulted in a lot of overtime spent developing the game. As this is only a 7.5 point course, we definitely spent too much time on it. (see repo for confirmation)


\unnumberedSection[graphicDesign]{Graphic Designing}
Restate that the fear came true.

\unnumberedSection[gpgsDowntime]{Google Play Service Downtime}
During our time developing the game, google play service had some issues where the service itself was unavailable. This caused a pretty significant delay in our development, as we were unable to test the implementation of the service. This seems to still be an ongoing issue, according to the forums for the service, so we chose not to publish the game since it works in its current state
