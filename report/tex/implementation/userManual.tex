\unnumberedChapter[userManual]{User Manual}

This chapter is destined to all the potential users of the game. We cover different way to install and run the application that would suit different type of users ranging from the traditional end-users to a future contributor. We will also shortly describe and explain the game interactive components. Screen captures will be used to support the speech.

\unnumberedSection[requirements]{Requirements}

\unnumberedSubsection[device]{Run on your Personal Device}
\begin{itemize}
	\item Make sure you have \textit{Google Play} installed
	\item Make sure you have the latest version of \textit{Google Play services} installed \newline
  Updating this application may be more difficult than expected. If you are experiencing any trouble, just follow the next steps. Unless you can skip to the next point.
	\begin{enumerate}
    \item Open \textit{Google Play Store}
    \item Search for \textit{Play Services \& Play Store Information} application
    \item Install the application
    \item Open the application
    \item Click on \textit{Play Store} button
    \item You should be redirected to the \textit{Google Play Store} installation page of the \textit{Google Play services}
    \item Finally, simply click to update
	\end{enumerate}
\end{itemize}

\unnumberedSubsection[androidStudioSetUp]{Android Studio Set Up}
\begin{itemize}
	\item Download \textit{Android Studio} %TODO: link
	\item Go to \texttt{Tools -> Android -> SDK Manager}
	\begin{itemize}
		\item https://prnt.sc/itxj13
		\item make sure the highlighted items are installed
	\end{itemize}
\end{itemize}

\unnumberedSubsection[emulator]{Run on an Emulator}
\begin{itemize}
	\item Make sure \textit{Android Studio} is set up has described in \nameref{subsec:androidStudioSetUp}.
	\item Install an emulator.
	\begin{itemize}
		\item http://prntscr.com/j8hvji
		\item Make sure the emulator has \textit{Google Play} installed.
		\item Choose the latest \gls{api} (27) for your device.
  \end{itemize}
  \item Follow the same steps for \nameref{subsec:device}
\end{itemize}

\unnumberedSection[installation]{Installation}
\begin{itemize}
	\item Install the provided \gls{apk} from the delivery.
	\item Upon application startup, you should be asked to log in to \textit{Google Play}. If you are already logged in on your device, this should be performed automatically for you.
	\item When the game starts you should see a prompt saying \texttt{successfull login}.
	\begin{itemize}
		\item If you get something the \texttt{error displaying signin 0} message, it must be because you are facing one of the following problems:
    \begin{itemize}
      \item Your \textit{Google Play service} is out of date
      \item You did not provide your email for the test version\footnote{The game is in closed alpha on the \textit{Google Play} console.}
      \item The login attempt did not succeed. \item Your phone does not have access to a proper Internet connection.
    \end{itemize}
	\end{itemize}
\end{itemize}

\unnumberedSection[play]{Play}
\begin{itemize}
	\item To start a game, simply press the “Find game” button.
  \item Then, an lobby creation activity should appear. During that time, \textit{Google Play services} \gls{api} will try to find opponents.
  \item Once the room is filled, you will be redirected to the game screen.
	\item You will then instantly be able to start playing.
\end{itemize}

\unnumberedSection[developer]{Building the Application from Source Code (Developer Mode)}
\begin{itemize}
	\item Make sure \textit{Android Studio} is set up has described in \nameref{subsec:androidStudioSetUp}.
	\item For you to be able to fully test the game, we need to make the \texttt{keystore}, that has been used to sign the build, match the one we have authorized in the \textit{Google Play} console.
	\begin{itemize}
		\item This key is required to make sure the application requesting the \gls{api} is authorized.
		\item This can be accomplished by using the \texttt{debug.keystore} we added to the delivery to sign the builds you create.
	\end{itemize}
	\item Build and run the game on a device that have followed the steps previously explained (see \nameref{subsec:device}).
\end{itemize}

\unnumberedSection[views]{Views}
Beside the activity automatically generated by \textit{Google Play services} \gls{api}, there are three main views to the game.
\begin{description}
	\item [Game Screen] The principal screen of the application which is rendered at game time. See the section \nameref{sec:gameplay} for more details.
	\item [Settings Screen] Screen allowing the user to modify general settings of the application such as the music volume and the sound effects volume.
	\item [Credits Screen] Screen listing all the copyright attribution with their corresponding asset. Even thought most of them were took for the public domain, we thought that it would still be interesting to thanks them for their work.
\end{description}

\unnumberedSection[gameplay]{Gameplay}
\unnumberedSubsection[controls]{Controls}
Once the game is started, the user can mainly interact with two controls which are the joystick and the bomb button. We put a lot of work to make those controls feel as natural as possible. \\

By moving the joystick around, the user will induce change in position of its character. At the moment the knob is released, its position is directly reseted to a default neutral position which will make the character stops. Depending on the relative position of the knob relatively to the center of the joystick, an orientation component will be extracted and send to the character to inform him that it has to modify its position according to this new orientation. Another interesting fact to observe is that the knob cannot fall off the joystick limits and that the joystick is fixed to the bottom left corner of the screen. \\

To plant a bomb, the user can simply press on the bottom right corner button displaying a bomb icon. Bare in mind that a cooldown is programmed over this button to avoid the unlimited spawning of bombs which would make the game less challenging and boring to play. However, by collecting \nameref{subsec:items}, a user can reduce this cooldown and get a competitive advantage over its opponents.

\unnumberedSubsection[obstacles]{Obstacles}
By planting bombs, as described in the previous section, a user can change the map configuration by destroying crates blocking its way. There are two different types of obstacles (walls, crates and bushes). As in the original \textit{Bomberman}, walls are indestructibles whereas crates and bushes can be destroyed using bombs.

\unnumberedSubsection[items]{Items}
By destroying crates, items may sometimes appear on the map. There are 6 types of collectible items. Their icons and associated effects are described below.
\begin{description}
\item [Heal Potions] Increase the maximum health by 25 points and regenerate 25 health points. The character health is initially set to 100 points and can increase up to 200 points. By after,
\item [Poison Potions] Reduce health by 50 points.
\item [Fire] Increase the damage inflected by the bombs you create by 50 points. Bombs inflect 10 points of damage by default. Again, the maximum amount of damage that a bomb can inflict is 200 points which means that it will instantly kill any player including yourself if you do not play carefully.
\item [Bomb] Reduce the cooldown over the bomb button by 0.1 second. The minimum cooldown if 0.5 second and it starts to 1 second. Considering that a bomb take 3 seconds to detonate, this imply that the user can spawn from 3 to 6 bombs at the time. Also take note that, if a bomb is in the range of another bomb, then this bomb will be detonated no matter the time left on its timer. Think fast! Some particular bomb positioning combined to the right timing can create deadly combos.
\item [Arrows] Increase the range of the explosion by 2 points without any defined limit. Highly ranged players may take advantage of getting out of sight from the opponents and easily take the escape by dropping bombs on their path.
\item [Wind] Increase the speed of the character.
\end{description}
